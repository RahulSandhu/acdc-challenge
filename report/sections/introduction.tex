\section{Introduction}

Advances in medical imaging and analytical techniques have long supported the
diagnosis and management of Cardiovascular Diseases (CVDs). The increasing
availability of high-resolution imaging and computational tools presents an
opportunity to enhance clinical decision-making through data-driven approaches.

\subsection{Motivation}

CVDs remain a major global health burden, causing nearly one-third of annual
deaths worldwide \cite{who2021cvd}. Early and reliable diagnosis is key to
managing progression and improving outcomes. Cardiac Magnetic Resonance (CMR)
is the gold standard for assessing heart morphology and function due to its
high resolution and non-invasive nature \cite{raisi2020cmr}. However,
traditional diagnostics often rely on limited parameters, potentially
overlooking patterns critical for certain conditions
\cite{lambin2017radiomics}.

\subsection{State of the Art}

Radiomics has recently emerged as a promising method to extract data from
medical images, providing a more detailed representation of tissue
characteristics. By quantifying traits like shape, texture, and intensity, it
enables the creation of datasets for automated disease classification. Combined
with Machine Learning (ML) models, these features offer new opportunities for
predictive tools beyond traditional image interpretation
\cite{gillies2016radiomics}.

\subsection{Objectives}

For this challenge, we have been provided with the Automated Cardiac Diagnosis
Challenge (ACDC) dataset \cite{bernard2018deep}, derived from cine-MRI scans
acquired at the University Hospital of Dijon (France). It offers standardized
imaging data capturing real-time heart movement across the cardiac cycle.
Radiomic features were extracted from these segmented heart structures using
the PyRadiomics library \cite{pyradiomics2024}. Each patient is assigned to one
of five diagnostic categories, making this a multi-class classification
problem. Therefore, the objective is to compare different supervised machine
learning algorithms and identify the best-performing model.

\subsection{Goals}

To address this objective, we define a set of specific goals that guide the
project workflow:

\begin{itemize}[label=\ding{51}]
	\item Perform an exploratory analysis to understand the data.
	\item Compare Simple Split and Stratified K-Fold techniques.
	\item Apply feature selection and dimensionality reduction.
	\item Implement ML models for supervised classification.
	\item Evaluate and compare model performance using robust metrics.
	\item Assess the clinical relevance of the best-performing model.
\end{itemize}
